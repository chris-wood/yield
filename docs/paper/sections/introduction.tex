\section{Introduction}
Today's media applications are increasingly data-centric. Online services
such as Netflix, Spotify, and Facebook all seek to provide data to their
clients as quickly as possible in diverse network and operating conditions. 
Serving this data has bandwidth, storage, and computation costs of varying
levels of expense. With respect to bandwidth, heterogeneous links long geographical 
distances, and rising congestion due to flash floods for data all contribute to
high expenses. Content Distribution Networks (CDNs) exist minimize this expense 
by moving data servers closer to clients at the edges
of the network. CDN points-of-presence (POPs) service fewer and more localized clients,
thereby improving at least two of these three problems. 

Content-Centric Networking (CCN) is a type of network architecture with one goal
of providing a native CDN to service many servers (data producers) and clients (consumers)
simultaneously. CCN serves to transfer named data from producers to consumers. 
Specifically, consumers obtain data by issuing a request for the desired data,
identified by its name. The network is responsible for locating the corresponding
data and returning the result to the consumer. Routers may optionally cache
this data in transit to service future requests for the same content. 

Compared to TCP/IP-based media services, CCN has several interesting advantages. 

-- no need to prime the caches
-- less protocol complexity 

XXX: talk about privacy problem and TLS everywhere (as another variant)

While this reduces the bandwidth bottleneck, it still does not solve the remaining
storage and computation problems. 

